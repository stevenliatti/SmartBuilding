\subsection{Bilan personnel}
Pour conclure, ce projet nous a permit d'apprendre et pratiquer Kafka, une technologie initaliemment inconnue, ce qui fût très instructif. Il nous à également permis d'approfondir les notions vue en cours concernant KNX, Openzwave et la detection des beacons depuis un appareil Android. En ajoutant a tout cela une partie devloppement Android, une base de données SQL et l'utilisation de python. Ce qui fait de ce projet un très bon cas pratique permettant de mettre en lien toutes ces technologies plus interessante les unes que les autres.
% TODO: Sucage de boule

\subsection{Problèmes rencontrés}
Les principales difficultés rencontrées durant le développement de ce projet sont les suivantes :
\begin{itemize}
    \item Docker python
    \item KNX la documentation fournie ne correspondait pas parfaitement avec le fonctionnement réeé du protocole
    \item Openzwave manque d'informations concernant le reset manuel du controller
\end{itemize}

% TODO: améliorations
\subsection{Améliorations possibles}
Les améliorations suivantes pourraient être apportées au projet :
\begin{itemize}
    \item
\end{itemize}