La technologie d'échange des messages était au choix, nous avons choisi d'expérimenter Kafka \cite{kafka}, étant donné qu'elle est de plus en plus répandue dans différents projets IoT, il était donc intéressant de la prendre en main au travers de ce projet.

\subsection{Généralités}
Apache Kafka est une plate-forme de diffusion d'événements distribuée open source, tolérante aux pannes, développée initialement par LinkedIn puis reprise par l'\textit{Apache Software Foundation}, écrite en Scala. Service de journalisation distribué, Kafka est utilisé en tant que \textit{hub} de messagerie haute performance, en raison de son débit, de son évolutivité, de sa fiabilité et de son mécanisme de réplication. Les messages Kafka peuvent être étiquetés et publiés/consommés dans un "sujet" (ou "\textit{topic}"). Chaque message est constitué d'une clé, d'une valeur et d'un \textit{timestamp}. Étant donné que Kafka est un système distribué, les rubriques peuvent être partitionnées et répliquées sur plusieurs noeuds. Les fonctionnalités phare de Kafka sont :
\begin{itemize}
    \item Publier et s'abonner à des flux de messages, similaires à une file d'attente de messages ou à un système de messagerie d'entreprise.
    \item Stocker les flux de messages d'une manière durable et tolérante aux pannes.
    \item Traiter les flux de messages au fur et à mesure qu'ils se produisent.
\end{itemize}

Kafka offre quatre \acrshort{api}s principales :
\begin{itemize}
    \item L'\acrshort{api} \textit{Producer} permet à une application de publier un flux de messages sur un ou plusieurs \textit{topics} Kafka.
    \item L'\acrshort{api} \textit{Consumer} permet à une application de s'abonner à un ou plusieurs \textit{topics} et de traiter le flux de messages produit.
    \item L'\acrshort{api} \textit{Streams} consommant un flux de messages à partir d'un ou plusieurs \textit{topics}, en leur appliquant un traitement, puis les reproduisant dans un ou plusieurs \textit{topics}.
    \item L'\acrshort{api} \textit{Connector} permet de construire des producteurs ou consommateurs Kafka réutilisables et connectés à des applications existantes. Par exemple un connecteur à une base de données qui capture tout changement dans celle-ci.
\end{itemize}

\subsection{Usage dans le projet}
Dans le cadre de ce projet, nous avons utilisé Kafka afin de communiquer entre les différentes entités de l'application.
Chacunes d'entre elles (KNX, OpenZWave, \acrshort{api} \acrshort{rest} Flask, DB translator et Automatic Controller) écoute et produit différents messages dans Kafka afin de réagir à certaines actions provenant du client et également de produire des informations à intervalles réguliers, sans dépendre de la demande des client. Au sein de ce projet, nous faisons usage de trois topics : 
\begin{enumerate}
    \item \mintinline{text}{knx} : concerne toutes les commandes KNX.
    \item \mintinline{text}{zwave} : concerne toutes les commandes OpenZWave.
    \item \mintinline{text}{db} : pour la production de la liste des \textit{devices} présents en base de données.
\end{enumerate}
