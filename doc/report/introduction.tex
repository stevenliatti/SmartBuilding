La forme "nous" est utilisée tout au long de ce rapport étant donné que ce projet est réalisé par binôme.

\subsection{Buts}
Le but de ce projet est de mettre en pratique les différentes librairies et technologies étudiées en cours d'internet des objets afin de réaliser un système de gestion de bâtiment intelligent qui offre des fonctionnalités de haut niveau (fonctionnalités pour l'utilisateur final) telles que :
\begin{itemize}
    \item Abaisser la température d'une pièce à un seuil donné lorsqu'elle est vide
    \item Augmenter la température d'une pièce à un seuil donné lorsqu'elle est occupée
    \item Fermer les stores lorsque l'humidité est élevée
    \item Ouvrir les stores le jour, lorsque la luminance est faible et que la pièce est occupée
    \item Afficher l'état d'un magasin et / ou d'un radiateur donné
    \item Surveiller manuellement les stores et les radiateurs de la pièce où se trouve l'utilisateur
    \item Fournir des statistiques
\end{itemize}
L'utilisateur intéragit avec le système à l'aide d'une application Android. Ces fonctionnalités peuvent être activées automatiquement ou manuellement.

\subsection{Motivations}
Le périmètre de ce projet semblait parfaitement adapté pour exploiter la majorité des techniques et méthodes vues durant le cours d'IoT. Ce projet est une formidable occasion pour relier la mise en pratique des connaissances acquises en cours et notre passion pour le développement d'un système complet multi couches (dit \textit{full stack}).

\subsection{Méthodologie de travail}
Sur la base d'une analyse préliminaire, nous avons séparé le travail en plusieurs tâches que nous avons assigné à chaque membre du binôme de manière équitable afin d'effectuer le travail en parallèle.
Nous avons adopté une pseudo méthode "agile", en factorisant le projet en petites tâches distinctes et en nous fixant des délais pour les réaliser. Le partage du code s'est fait avec git et gitlab. Nous nous sommes servis des \textit{issues} gitlab pour représenter nos tâches et du "\textit{board}" du projet pour avoir une vision globale du travail accompli (par qui et quand) et du travail restant.
