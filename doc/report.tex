\documentclass[a4paper, 10pt]{article}
\usepackage[utf8]{inputenc}
\renewcommand\familydefault{\sfdefault}
\usepackage[T1]{fontenc}
\usepackage[francais]{babel}
\usepackage[left=2.5cm,top=2.5cm,right=2.5cm,bottom=2.5cm]{geometry}
\usepackage[onehalfspacing]{setspace}
\usepackage{graphicx}
\usepackage[usenames, dvipsnames]{xcolor}
\definecolor{mygray}{gray}{0.95}

\usepackage{minted}
\usemintedstyle{colorful}
\usepackage{float}
\floatplacement{figure}{H}
\usepackage{authblk}
\usepackage{enumitem}
\setlist[enumerate]{label*=\arabic*.}
\usepackage{hyperref}
\hypersetup{
    colorlinks,
    citecolor=black,
    filecolor=black,
    linkcolor=black,
    urlcolor=blue
}

\usepackage{caption}
\newenvironment{code}{\captionsetup{type=listing}}{}
\usepackage{array}
\usepackage{etoolbox}
\patchcmd{\thebibliography}{\section*{\refname}}{}{}{}

\usepackage{dirtree}
\usepackage{tabularx}

\usepackage{glossaries}
	\let\oldnewacronym\newacronym
	\newcommand*{\provideacronym}[3]{%
	  \ifglsentryexists{#1}{%
	  }{%
	    \oldnewacronym{#1}{#2}{#3}%
	  }%
	}
\makeglossaries

\usepackage{fancyhdr}
\pagestyle{fancy}
\fancyhf{}
\lhead{Projet IoT}
\rhead{\leftmark}
\lfoot{Jeremy Favre - Steven Liatti}
\rfoot{\thepage}
\renewcommand{\footrulewidth}{1pt}

\usepackage{pdfpages}

\begin{document}

\title{Projet IoT}
\author{Jeremy Favre - Steven Liatti}
\date{Janvier 2020}
\maketitle
\begin{figure}
    \begin{center}
        \includegraphics[width=0.65\textwidth]{img/app_framed.png}
    \end{center}
\end{figure}
\newpage


% \setcounter{tocdepth}{2}
\addcontentsline{toc}{subsection}{Table des matières}
\tableofcontents
\addcontentsline{toc}{subsection}{Table des figures}
\listoffigures
% \renewcommand{\listtablename}{Table des tables}
% \addcontentsline{toc}{subsection}{Table des tables}
% \listoftables
\newpage
\renewcommand\listoflistingscaption{Table des listings de code source}
\addcontentsline{toc}{subsection}{Table des listings de code source}
\listoflistings

\subsection*{Conventions typographiques} %-----------------------------------------------------------------------------------------------
\addcontentsline{toc}{subsection}{Conventions typographiques}
Lors de la rédaction de ce document, les conventions typographiques ci-dessous ont
été adoptées.
\begin{itemize}[label=\textbullet]
	\item Tous les mots empruntés à la langue anglaise ou latine ont été écrits en \textit{italique}.
	\item Toute référence à un nom de fichier (ou répertoire), un chemin d'accès, une 
    utilisation de paramètre, variable, commande utilisable par l'utilisateur, ou extrait de code 
    source est écrite avec une police d'écriture à \mintinline{text}{chasse fixe}.
	\item Tout extrait de fichier ou de code est écrit selon le format suivant:
    \bigbreak
    \begin{code}
        \begin{minted}[bgcolor=mygray,breaklines,breaksymbol=,linenos,frame=single,stepnumber=1,tabsize=2]{kotlin}
fun main() {
    println("Hello World!")
}
        \end{minted}
    \end{code}
\end{itemize}


\newacronym{api}{API}{\textit{Application Programming Interface}, Interface de programmation : services offerts par un programme producteur à d'autres programmes consommateurs}
\newacronym{json}{JSON}{\textit{JavaScript Object Notation}, Format d'échange de données léger, facile à lire et écrire par les humains et les machines}
\newacronym{uri}{URI}{\textit{Uniform Resource Identifier}, Identifiant uniforme de ressource, une courte chaîne de caractères identifiant une ressource sur un réseau}
\newacronym{rest}{REST}{\textit{representational state transfer}, style d'architecture logicielle définissant un ensemble de contraintes à utiliser pour créer des services web}
% \newacronym{xml}{XML}{\textit{Extensible Markup Language}, Langage extensible de description de données}
% \newacronym{jvm}{JVM}{\textit{Java Virtual Machine}, Exécute le bytecode Java sur différents systèmes d'exploitation}
\printglossary[type=\acronymtype,title={Acronymes}]
\newpage


\section{Introduction} %-----------------------------------------------------------------------------------------------
Ce projet concerne le dévelopement d'un logiciel de construction moderne qui offre des fonctionnalités de haut niveau (fonctionnalités pour l'utilisateur final) telles que :
\begin{itemize}
    \item Abaisser la température d'une pièce à un seuil donné lorsqu'elle est vide
    \item Augmenter la température d'une pièce à un seuil donné lorsqu'elle est occupée
    \item Fermer les stores lorsque l'humidité est élevée
    \item Ouvrir les stores le jour, lorsque la luminance est faible et que la pièce est occupée
    \item Afficher l'état d'un magasin et / ou d'un radiateur donné
    \item Surveiller manuellement les stores et les radiateurs de la pièce où se trouve l'utilisateur
    \item Fournir des statistiques
\end{itemize}
Ces fonctionnalités peuvent être activées automatiquement ou manuellement. La forme "nous" est utilisée tout au long de ce rapport étant donné que ce projet est réalisé par binôme.

\subsection{Buts}
Le but de cette application est de mettre en pratique les différentes librairies et tecchnologies étudiées en cours d'internet des objets afin de produire une interafce utilisateur déployée sur un smartphone Android.

\subsection{Motivations}
Le périmètre de ce projet semblait parfaitement adapté pour exploiter la majorité des techniques et méthodes vues durant le cours d'IOT. Ce projet est une formidable occasion pour relier la mise en pratique des connaissances acquises en cours et notre passion pour le dévloppement full stack.

\subsection{Méthodologie de travail}
Sur la base d'une analyse préliminaire, nous avons séparé le travail en plusieurs tâches que nous avons assigné à chaque membre du binôme de manière équitable afin d'effectuer le travail en parallèle.
Nous avons adopté une pseudo méthode "agile", en factorisant le projet en petites tâches distinctes et en nous fixant des délais pour les réaliser. Le partage du code s'est fait avec git et gitlab. Nous nous sommes servis des \textit{issues} gitlab pour représenter nos tâches et du "board" du projet pour avoir une vision globale du travail accompli (par qui et quand) et du travail restant.

% TODO: remove
\cite{tmdb} \cite{book}
\newpage

\section{Conception et analyse} %-----------------------------------------------------------------------------------------------
Notre système s'architecture autour du \textit{message broker} Kafka \cite{kafka}, visible à la figure \ref{shema_general}. Les contrôleurs KNX et Openzwave gèrent respectivement les stores et les radiateurs ainsi que les lumières et capteurs multi-fonctions (présence, luminance, température, humidité). Ils peuvent d'une part recueillir les informations des \textit{devices} associés, mais également attribuer des nouvelles valeurs pour chacuns. Un utilisateur, muni de l'application mobile et se trouvant à portée d'un Beacon pourra intéragir avec les \textit{devices} de la pièce (\textit{room}) associée avec le Beacon. Ces interactions comportent la visualisation de l'état actuel des \textit{devices} (présence, niveau de température, d'humidité et de lumière, ouverture des stores et des radiateurs) mais aussi leur contrôle (lumière, stores et radiateurs). L'application échangera avec un serveur HTTP REST qui fera office de \textit{backend}, lisant dans la base de données les relations entre Beacons, pièces, devices et utilisateurs. La base de données garde également en mémoire les données des \textit{devices}, pour des éventuelles statistiques.

\begin{figure}
    \begin{center}
        \includegraphics[width=0.8\textwidth]{img/general.png}
    \end{center}
    \caption{Architecture globale du système}
    \label{shema_general}
\end{figure}

\newpage

\section{Kafka} %-----------------------------------------------------------------------------------------------
La technologie d'échange des messages était au choix, nous avons choisi d'expérimenter Kafka \cite{kafka}, étant donné qu'elle est de plus en plus répandue dans différents projets IoT, il était donc intéressant de la prendre en main au travers de ce projet.

\subsection{Généralités}
Apache Kafka est une plate-forme de diffusion d'événements distribuée open source, tolérante aux pannes, développée initialement par LinkedIn par l'\textit{Apache Software Foundation}, écrit en Scala. Service de journalisation distribué, Kafka est utilisé en tant que \textit{hub} de messagerie haute performance, en raison de son débit, de son évolutivité, de sa fiabilité et de son mécanisme de réplication. Les messages Kafka peuvent être étiquetés et publiés/consommés dans un "sujet" (ou "\textit{topic}). Chaque message est constitué d'une clé, d'une valeur et d'un \textit{timestamp}. Étant donné que Kafka est un système distribué, les rubriques peuvent être partitionnées et répliquées sur plusieurs noeuds. Les fonctionnalités phare de Kafka sont :
\begin{itemize}
    \item Publier et s'abonner à des flux de messages, similaires à une file d'attente de messages ou à un système de messagerie d'entreprise.
    \item Stocker les flux de messages d'une manière durable et tolérante aux pannes.
    \item Traiter les flux de messages au fur et à mesure qu'ils se produisent.
\end{itemize}

Kafka offre quatre \acrshort{api}s principales :
\begin{itemize}
    \item L'\acrshort{api} \textit{Producer} permet à une application de publier un flux de messages sur un ou plusieurs \textit{topics} Kafka.
    \item L'\acrshort{api} \textit{Consumer} permet à une application de s'abonner à un ou plusieurs \textit{topics} et de traiter le flux de messages produit.
    \item L'\acrshort{api} \textit{Streams} consommant un flux de messages à partir d'un ou plusieurs \textit{topics}, en leur appliquant un traitement, puis les reproduire dans un ou plusieurs \textit{topics}.
    \item L'\acrshort{api} \textit{Connector} permet de construire des producteurs ou consommateurs Kafka réutilisables et connectés à des applications existantes. Par exemple un connecteur à une base de données qui capture tout changement dans celle-ci.
\end{itemize}

\subsection{Usage dans le projet}
Dans le cadre de ce projet, nous avons utilisé Kafka afin de communiquer entre les différentes entités de l'application.
Chacunes d'entre elles (KNX, OpenZWave, \acrshort{api} \acrshort{rest} Flask, DB translator et Automatic Controller) écoute et produit différents messages dans Kafka afin de réagir à certaines actions provenant du client et également de produire des informations à intervalles réguliers, sans dépendre de la demande des client. Au sein de ce projet, nous faisons usage de trois topics : 
\begin{enumerate}
    \item \mintinline{text}{knx} : concerne toutes les commandes KNX.
    \item \mintinline{text}{zwave} : concerne toutes les commandes OpenZWave.
    \item \mintinline{text}{db} : pour la production de la liste des devices présents en base de données.
\end{enumerate}

\newpage

\section{Implémentation} %-----------------------------------------------------------------------------------------------
\subsection{Android}
Comme nous l'avons mentionné dans les rubriques précédentes, le client à été developpé pour Android. Pour cela, nous avons choisi d'utiliser le langage Kotlin qui est plus preformant et optimise les opérations et la structure en opposition à Java. C'est également le langage que nous utilisons au sein du cours d'Android, c'est donc un bon moyen de mettre en relation les deux cours.

Le rôle principal de ce client est de détecter (grâce au bluetooth) le beacoin avec la plus grande proximité. En fonction de ce beacoin le client peut envoyer une requête permettant de determiner la salle dans laquel se trouve la personne et les devices qui y ont présentes (stors, radiateurs, lumières etc.).Une fois l'emplacement determiné et les devices détéctés, les contrôles et les informations des différents devies apparaissent sur l'interface graphique de l'utilisateur, lui permettant de contrôller ou obtenir les différentes informations des devices à proximité. 

\subsection{Broker Kafka}
Le broker Kafka joue le rôle de coordinateur entre tous les différents clients qui consomment et produisent des messages dans les différents topics.
Nous avons choisi de déployer ce Broker Kafka sur une instance AWS permettant de le rendre accessible par toutes les entités de l'application.
Nous avons également associé un nom de domaine à cette instace ce qui permet de la référencer de manière plus lisible et agérable par les clients.

\subsection{Rest server Flask}
Etant donné qu'il n'existe pas enciore de librairie permettant d'utiliser dirctement le client Kafka sur Android, nous avons du mettre en place un adapter que nous permet de faire la translation entre Kafka Android.
Pour ce faire nous avons utilisé la librairie Flask de Python 3 qui permet de mettre en place une API REST.
Cette API joue également le rôle de producteur et de consumateur Kafka afin de mettre en relation les différentes entités.
D'une part, le consomateur Kafka se charge de consommer les différents messages de KNX et Openzwave afin de stocker les informations utiles à l'application et aux statistiques dans la base de données.
D'autre part, nous avons le producteur Kafka qui se charge de transformer les actions reçues sous forme de requête HTTP en message Kafka envoyé directement dans le bon topic ce qui permet donc d'effectuer les actions demandées en conséquence (ouverture des stors par exemple).

\subsubsection{Routes exposées}
Voici la liste exhaustive des différentes routes mise à disposition par notre API REST : 
% TODO: prendre les routes et bien formatter


\subsection{KNX lib}
En se basant sur le code de l'exercice KNX réalisé au cours du premier service, nous l'avons adapté de manière à en faire une librairie réutilisabe depuis les producteur et consomateur Kafka.
Cette libraire contient une classe \mintinline{python}{knx} qui se charge de créer la connexion avec knx dans son constructeur et expose les différentes méthodes relatives à l'utilisation des devies KNX.
Voici les méthodes intégrées à notre libraire KNX :
\begin{itemize}
    \item \mintinline{python}{send_datas(self, group_address, data, data_size, apci, read=False)} : cette méthode est utilisée afin de transmettre des informations à KNX elle recoit différents arguments variables qui permettent de controller/lire les informations des différentes devices en adaptant leurs emplacement qui sont identifiable à l'aide de l'étage et de la salle (voir protocole KNX).
    \item \mintinline{python}{disconnect(self)} => permet de se deconnecter de KNX
\end{itemize}

\subsection{Zwave lib}
Nous avons réutilisé la librairie que nous avions développée durant le premier semestre qui comprend les différentes méthodes permettnt d'intéragir avec le réseau Openzwave. 
Dans le cadre de ce projet, nous n'utilisons plus cette libraire au travers d'un serveur Flask, mais directemnt dans le producteur et le consommateur Kafka.
Tout cela nous permet donc d'appliquer différentes actions qui ont été consummée et également de produire à intervalles réguliers les informations émise par les devices de notre salle.

% TODO: revoir
\subsection{Adapter Kafka OUUU ?????? Protocole des messages}
En ce qui concerne les messages consommés par KNX et openzwave, nous avons choisi de définir notre propre protocol.
Celui ci fait correspondre la clé du message reçu à l'action à effectuer et le contenu du message aux éventuels paramètres à transmettre. Ces différents paramètres sont au format JSON puis encodés en bytes afin d'être transmits au broker Kafka consumés par une autre entité.

\subsubsection{KNX}
Pour KNX nous retrouvons les messages suivants : 
\paragraph{Production}
\begin{itemize}
    \item \mintinline{python}{read_percentage_blinds} : Ce messages est produit à intervalles de 5 secondes, il permet d'envoyer dans le topic \mintinline{text}{KNX} le pourcentage d'ouverture de tous les stores de toutes les salles. 
\end{itemize}

\paragraph{Consommation}
    L'étage et la chambre permettent d'identifier l'appareil sur lequel il faut agir, pour cela ces deux paramètres sont donnés dans le corps du message ce qui permet d'intéragir avec le bon device en fonction de la position de l'utilisateur.
\begin{itemize}
    \item \mintinline{python}{open_blinds} : Permet d'ouvrir les stores (100\%)
    \item \mintinline{python}{close_blinds} : Permet de fermer les stores (0\%)
    \item \mintinline{python}{percentage_blinds} : Met les stores à un certain pourcentage qui est passé dans la valeur du message
    \item \mintinline{python}{percentage_radiator} : Met les radiateurs à un certain pourcentage qui est passé dans la valeur du message
\end{itemize}

\subsubsection{Openzwave}
Openzwave est capable de traiter les messages suivants : 
\paragraph{Production}

\paragraph{Consommation}

\subsection{Base de données}
En ce qui concerne la base de données, nous avons opté pour une base relationnelle avec MySQL qui nous a permis de représenter les différentes structures de données et les mettre en lien facilement et de manière efficace.

Les tables de la base nous permettent de persister les données telles que les devices KNX, les noeuds openzwave, les identifiants des beacons et leurs association avec une room spécifique. Deux tables sont également dediées au stockage des logs pour knx et openzwave afin de pouvoir retracer les données dans le temps et tirer des statistiques.

Voici le diagramme représentant notre base de données :


\begin{figure}
    \begin{center}
        \includegraphics[width=0.8\textwidth]{img/diagramme_db.jpg}
    \end{center}
    \caption{Schéma relationnel de la base de données}
    \label{db_schema}
\end{figure}

\newpage

\section{Conclusion} %-----------------------------------------------------------------------------------------------
% TODO: relire tout le fichier
\subsection{Bilan personnel}
Pour conclure, ce projet nous a permis d'apprendre et pratiquer Kafka, une technologie initialement inconnue pour nous, ce qui fût très instructif. Il nous a également permis d'approfondir les notions vues en cours concernant KNX, Openzwave et la détection des Beacons depuis un appareil Android. En ajoutant à tout cela une partie développement Android, une base de données SQL et l'utilisation de python. Ce projet est un très bon cas pratique mettant en lien toutes ces technologies plus intéressantes les unes que les autres.

\subsection{Problèmes rencontrés}
Néanmoins, plusieurs difficultés ont été rencontrées durant le développement de ce projet :
\begin{itemize}
    \item Toute l'architecture Kafka est déployée avec docker-compose. Afin de configurer le tout de manière automatique, nous avons rencontrés quelques problèmes, notamment liés à la sortie standard des containers basés sur l'image python. En effet, le container n'affiche pas la sortie standard des scripts python dans les logs si une option spécifique n'est pas renseignée dans le \mintinline{text}{docker-compose.yml}. Pendant un bon moment nous croyions que le consommateur ne fonctionnait pas, alors que c'était uniquement l'affichage qui était défaillant.
    \item En ce qui concerne KNX, la documentation fournie ne correspondait pas parfaitement avec le fonctionnement réel du protocole, notamment au niveau des trames renvoyées après la lecture/écriture des données dans le socket. Ce protocole est également peu pratique lorsqu'on souhaite l'utiliser pour transmettre plusieurs messages consécutifs sans fermer la connexion.
    \item La procédure d'utilisation fournie pour Openzwave manque d'informations concernant le reset manuel du \textit{controller}. La documentation officielle manque également cruellement de précision pour la plupart méthodes.
    \item De manière générale, l'utilisation de python pour ce genre de système distribué est à notre avis peu pratique, python n'étant pas typé et compilé, toutes les cas ont dus être massivement testés à l'exécution pour être certains que le code marche. De plus, dans un contexte de sérialisation et désérialisation fréquente, il est plus judicieux et pratique de disposer d'une conversion fortement typée des données.
\end{itemize}

\subsection{Améliorations possibles}
Les améliorations suivantes pourraient être apportées au projet :
\begin{itemize}
    \item Gestion des droits des utilisateurs : chaque utilisateur devrait pouvoir s'authentifier dans l'application à l'aide d'un \textit{username/password}, stockés en base de données. Chaque utilisateur aurait également une liste des salles associées à son profil. Etant donné que l'application détecte la salle à l'aide du Beacon le plus proche, il serait alors très simple de savoir si un utilisateur dispose des droits nécessaires pour accéder aux fonctionnalités proposées par les \textit{devices} de la salle ou il se trouve.
    \item Perfectionnement de l'expérience utilisateur de l'application Android.
\end{itemize}

\newpage

\section{Références} %-----------------------------------------------------------------------------------------------
\bibliographystyle{unsrt}
\bibliography{bib}

\end{document}

% \begin{figure}
%     \begin{center}
%         \includegraphics[width=0.8\textwidth]{images/image.png}
%     \end{center}
%     \caption{légende}
%     \label{label}
% \end{figure}

% \bigbreak
% \begin{code}
%     \begin{minted}[bgcolor=mygray,breaklines,breaksymbol=,linenos,frame=single,stepnumber=1,tabsize=2]{rust}
% fn main() {
%     println!("Hello, world!");
% }
%     \end{minted}
%     \caption{Hello world en Rust}
%     \label{label}
% \end{code}
% \bigbreak

% \bigbreak
% \begin{code}
%     \inputminted[bgcolor=mygray,breaklines,breaksymbol=,linenos,frame=single,stepnumber=1,
%         tabsize=2,firstline=157,lastline=185]{rust}{file.rs}
%     \caption{légende}
%     \label{label}
% \end{code}
% \bigbreak
